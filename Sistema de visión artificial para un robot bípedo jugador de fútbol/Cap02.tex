\chapter{Marco teórico}
\section{Robots bípedos}
Un robot bípedo es un tipo de robot móvil que cuenta con dos piernas, y su principal diferencia entre otros es su locomoción bípeda. \cite{yang2017stateart} El ejemplo más común y concreto de esta clasificación son los robots humanoides. La investigación en este campo ha tenido lugar desde la década de 1960, cuando RSmo-sher de American General Corporation produjo el primer robot de caminata bípeda, Rig. Este hito marcó el inicio de la investigación en robots humanoides, siendo el científico yugoslavo M. Vukobratovic quien propuso en 1969 la base para esta área: el criterio de estabilidad Zero Moment Point (ZMP).\cite{chen2013walking}
\section{Conceptos básicos de visión artificial} 

\section{Redes neuronales artificiales}
    
\section{Competencia Robocup Humanoid kid size}
