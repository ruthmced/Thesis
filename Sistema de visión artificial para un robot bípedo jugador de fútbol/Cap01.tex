\chapter{Introducción}
%contextualizar: uso de robots móviles haaciendo ref a los que tienen mucho potencial humanoides. El arg. de usarlos es que puede moverse en cualquier ambiente porque no usa ruedas.
 Los robots son cada vez más utilizados en ambientes de trabajo y vida cotidiana.\cite{sadangharn2022acceptance} Una de las clasificaciones que destaca son los robots móviles debido a sus habilidades y aplicaciones. En este mismo contexto, uno de los más importantes son los robots humanoides, que, al no usar ruedas pueden desenvolverse en diversos ambientes y a su vez reproducir capacidades humanas, como caminar. \cite{rubio2019review} 
\section{Motivación}
%primer idea: En los robots móviles la visión artificial es una de las habilidades más requeridas (robs de serv, de rescate, conduct,...)
%segunda idea: para fomentar el desarrollo de robots humanoides existen competencias como la robocup, donde se busca que un robot juegue fútbol para lo que se requieren de varias habilidades.
%Se debe mencionar que la visión es un problema abierto..
\section{Planteamiento del problema}
%se requiere hace un sist de visión que reconozca porterías y otros jugadores usando únicamnete cámaras rgb (se requiere probar en simulación y con un robot real).
\section{Hipótesis}

\section{Objetivos}
\subsection*{Objetivo general}
Desarrollar un sistema de visión artificial para identificar porterías y jugadores en un partido de fútbol con otros robots bípedos.
\subsection*{Objetivos particulares}
\begin{itemize}
    \item Aplicar características geométricas para identificar y localizar porterías.
    \item Aplicar redes neuronales artificiales para la detección de otros humanoides en la cancha.
    \item Evaluar el desempeño en un ambiente simulado.
    \item Evaluar el desempeño en un ambiente real.
\end{itemize}
\section{Descripción del documento}
