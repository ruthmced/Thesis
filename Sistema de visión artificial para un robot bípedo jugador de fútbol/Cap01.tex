\chapter{Introducción}
%contextualizar: uso de robots móviles haaciendo ref a los que tienen mucho potencial humanoides. El arg. de usarlos es que puede moverse en cualquier ambiente porque no usa ruedas.
La robótica ha sido testigo de avances significativos a lo largo de los años. Comenzó con robots industriales que realizaban tareas repetitivas en ambientes muy controlados. Hoy, los robots han evolucionado a ser más inteligentes y versátiles\cite{rayhan2023artificial}, lo que los hace ser cada vez más utilizados en ambientes de trabajo y vida cotidiana.\cite{sadangharn2022acceptance} \\Una de las clasificaciones que destaca son los robots móviles debido a sus habilidades y aplicaciones. En este mismo contexto, uno de los más importantes son los robots humanoides.\cite{rubio2019review} La importancia de esta clasificación recae en sus características, que pueden ser reducidas a 3; los robots humanoides son capaces de desenvolverse en diversos ambientes, esto permite que no sea necesario alterar el ambiente de trabajo humano y resultaría más económico modificar el robot que el ambiente completo, pueden usar herramientas como los humanos y su forma es similar.\cite{kajita2014introduction}
 Lograr que un robot desempeñe las actividades que realiza un ser humano, como caminar la percepción de objetos ha significado un reto, ya que, para poder decir que una máquina piensa como un humano, debe haber una manera de poder  determinarlo.\\La inteligencia artificial como un nuevo campo de estudio en la ciencia e ingeniería, se ha desarrollado en cuatro aproximaciones divididas en actuar y pensar humana, y racionalmente. La tarea de percibir y manipular objetos se encuentra dentro del actuar humanamente, donde se involucran las dos disciplinas de interés en este documento: robótica y visión computacional. \cite {russell2016artificial}\\ La visión es el sentido más poderoso, ya que a través de él se obtiene la información del ambiente y con ello se puede interactuar inteligentemente con el entorno.\cite{milella2006computer} La visión computacional es la extracción automatizada de información que provee una imagen. Dicha información puede tratarse de modelos 3D, la posición de una cámara, reconocimiento y detección de objetos.\cite{erik2012programming} Con el paso del tiempo los robots móviles tendrán que explorar cada vez distancias más grandes, por lo que será primordial contar con sensores avanzados y buena capacidad de percepción. 

\section{Motivación}
%primer idea: En los robots móviles la visión artificial es una de las habilidades más requeridas (robs de serv, de rescate, conduct,...)
Una de las tareas más importantes de un sistema autónomo de cualquier tipo es adquirir conocimiento de su entorno\cite{siegwart2011introduction} . Por ende, la visión artificial es una de las habilidades más requeridas en los robots móviles como robots de servicio, de rescate o conductores.
%segunda idea: para fomentar el desarrollo de robots humanoides existen competencias como la robocup, donde se busca que un robot juegue fútbol para lo que se requieren de varias habilidades.
\\
Para fomentar el desarrollo de robots humanoides existen competencias como la RoboCup. En la categoría de humanoid kid size se busca  que un robot juegue fútbol en un entorno dinámico, para lo que se requieren varias habilidades. Entre muchas otras cosas, el robot debe percibir y comprender su entorno visual para jugar de una manera efectiva. Con estas competencias lo que se busca es desafiar, probar e impulsar la investigación constante en nuevos algoritmos cada vez más eficientes. \cite{fiedler2019open} \\Este tipo de desafíos no dejan de ser parte de una investigación activa, ya que el campo de la visión computacional sigue representando un problema abierto. Puesto que se parte de información insufciente para dar solución a incógnitas que se quieren resolver, se usan modelos probabilíticos o inteligencia artificial, con el objetivo de eliminar la ambigüedad entre posibles respuestas. Sin embargo, esto sigue siendo muy propenso a fallar.\\ En este contexto, es conveniente partir del problema hacia la técnica adecuada, es decir, darle un enfoque ingienieril. \cite{szeliski2022computer}
%Se debe mencionar que la visión es un problema abierto..

\section{Planteamiento del problema}
%se requiere hace un sist de visión que reconozca porterías y otros jugadores usando únicamnete cámaras rgb (se requiere probar en simulación y con un robot real).

\section{Hipótesis}

\section{Objetivos}
\subsection*{Objetivo general}
Desarrollar un sistema de visión artificial para identificar porterías y jugadores en un partido de fútbol con otros robots bípedos.
\subsection*{Objetivos particulares}
\begin{itemize}
    \item Aplicar características geométricas para identificar y localizar porterías.
    \item Aplicar redes neuronales artificiales para la detección de otros humanoides en la cancha.
    \item Evaluar el desempeño en un ambiente simulado.
    \item Evaluar el desempeño en un ambiente real.
\end{itemize}
\section{Descripción del documento}
